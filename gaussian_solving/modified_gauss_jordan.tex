We can introduce a matrix notation for such equation set, with exponents of parameters as values:
\begin{equation}
\begin{bmatrix}[cccc|c]
    1 & 1 & 1 & 1 & c_1\\
    0 & 0 & 1 & 1 & c_2\\
    1 & 1 & 0 & 1 & c_3\\
    0 & 0 & 1 & 0 & c_4\\
\end{bmatrix}
\end{equation}
Elementary row operation in standard Gauss--Jordan algorithm consist of 
\begin{enumerate}
    \item Row switching
    \item Row addition (denoted as $\oplus$)
    \item Row multiplication by a non-zero constant (denoted as $\odot$)
\end{enumerate}
\textbf{Row switching} does not change in our modified Gauss-Jordan algorithm.

\textbf{Row addition} in regular Gauss--Jordan elimination is equivalent to adding coefficients of corresponding parameters.
In order to preserve the same result in matrix notation (which in our case stores information about the exponents rather than coefficients), we multiply both sides of equations:
\begin{equation}
    \begin{bmatrix}[ccc|c]
        1 & 0 & 2 & b_1 \\
        -1 & 2 & 0 & b_2 \\
    \end{bmatrix}
    \begin{matrix}
        r_1 = r_1 \oplus r_2 \\
        &
    \end{matrix}
    \sim
    \begin{bmatrix}[ccc|c]
        0 & 2 & 2 & b_1 \oplus b_2 \\
        -1 & 2 & 0 & b_2 \\
    \end{bmatrix}
\end{equation}

\begin{equation}
    \begin{cases}
        q_1 \cdot q_3^2 &= b_1 \\
        q_1^{-1} \cdot q_2^2 &= b_2 \\
    \end{cases}
    \begin{matrix}[c|c]
        & r1 = r1 \cdot r2 \\
        & \\
    \end{matrix} \sim
    \begin{cases}
        q_2^2 \cdot q_3^{2} &= b_1 \cdot b_2 \\
        q_1^{-1} \cdot q_2^2 &= b_2 \\
    \end{cases}
\end{equation}

\textbf{Row multiplication by a non-zero constant} was previously equivalent to multiplication of each coefficient by a constant $d$.
In order to multiply each exponent of each parameter, we simply raise both sides of equation to the power $d$.
\begin{equation}
    \begin{bmatrix}[ccc|c]
        1 & 0 & 2 & b_1 \\
        -1 & 2 & 0 & b_2 \\
    \end{bmatrix}
    \begin{matrix}
        r_1 = r_1\odot \frac{1}{3} \\
        r_2 = r_2\odot (-3)
    \end{matrix}
    \sim
    \begin{bmatrix}[ccc|c]
        \frac{1}{3} & 0 & \frac{2}{3} & b_1 \odot \frac{1}{3} \\
        3 & -6 & 0 & b_2 \odot (-3) \\
    \end{bmatrix}
\end{equation}
\begin{equation}
    \begin{cases}
        q_1 \cdot q_3^2 &= b_1 \\
        q_1^{-1} \cdot q_2^{2} &= b_2
    \end{cases}
    \begin{matrix}[c|c]
        & r_1 = r_1 ^ {\;\frac{1}{3}} \\
        & r_2 = r_2 ^ {-3}
    \end{matrix}
    \sim
    \begin{cases}
        q_1^{\frac{1}{3}} \cdot q_3^{\frac{2}{3}} &= b_1^{\frac{1}{3}} \\
        q_1^{3} \cdot q_2^{-6} &= b_2^{-3}
    \end{cases}
\end{equation}

This way we can define a new set of elementary operations as described in Table~\ref{tab:modified_elementary}.
\begin{table}[h!]
\begin{center}
\begin{tabular}{|l|p{7cm}|}
    \hline
    Gauss--Jordan elimination & Modified Gauss--Jordan elimination \\\hline\hline
    Row switching & Row switching \\\hline
    Row addition & Row multiplication \\\hline
    Row multiplication by a non-zero constant & Row exponentiation by a non-zero constant \\\hline
\end{tabular}
	    \caption{Elementary operations of modified Gauss--Jordan elimination}
		\label{tab:modified_elementary}
\end{center}
\end{table}
